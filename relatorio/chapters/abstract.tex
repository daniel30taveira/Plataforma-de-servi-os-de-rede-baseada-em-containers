The main objective of this project is the design and implementation of a network services platform managed in a reproducible way, using exclusively open-source tools. The developed infrastructure encompasses fundamental services for a small to medium-sized network, such as a DNS server, HTTPS Reverse Proxy, web services, e-mail, file server, and a VPN.

In contrast to traditional monolithic approaches, all services are implemented in containers (Docker/Podman) through declarative configuration, ensuring environment versioning and reproducibility. The work focuses on the strict application of security best practices, including the principle of least privilege, service hardening, and TLS encryption. Additionally, the project integrates advanced observability mechanisms (for metrics and centralized logs) and deployment orchestration using tools like Ansible and CI/CD pipelines. The final result demonstrates how a small data center can be managed in an automated, agile, and fault-resilient manner.

\vspace{1em}
\noindent \textbf{Keywords:} Containers, Orchestration, Network Services, CI/CD, Observability, Open Source.