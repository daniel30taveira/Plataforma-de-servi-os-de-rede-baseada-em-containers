\chapter{Introdução}
\label{chap:introducao}

A administração de sistemas e redes tem sofrido uma evolução drástica nos últimos anos. A necessidade de implementar e gerir serviços de forma rápida, segura e sem interrupções impulsionou a adoção de tecnologias de \textit{containerização} e práticas de automação de infraestruturas.

\section{Contextualização e Motivação}
Numa infraestrutura tecnológica de pequena ou média dimensão, a gestão manual de múltiplos serviços (como DNS, \textit{web}, \textit{proxies} e VPNs) torna-se insustentável a longo prazo, sendo propensa a erros humanos e dificultando a recuperação em caso de falhas. A adoção do paradigma de "Infraestrutura como Código" (IaC) e de arquiteturas baseadas em \textit{containers} permite mitigar estes problemas, oferecendo ambientes isolados, padronizados e facilmente replicáveis.

\section{Objetivos}
O objetivo central deste projeto é construir uma plataforma de serviços de rede baseada em \textit{containers} e orquestração leve, focada em boas práticas de segurança, CI/CD (\textit{Continuous Integration / Continuous Deployment}) e observabilidade. 

De forma mais específica, pretende-se:
\begin{itemize}
    \item Implementar serviços típicos de rede (DNS, \textit{Reverse Proxy} HTTPS, \textit{Web}, E-mail, Servidor de Ficheiros e VPN) em \textit{containers} (Docker/Podman).
    \item Demonstrar a gestão de um pequeno \textit{data center} de forma totalmente reprodutível utilizando apenas ferramentas \textit{open source}.
    \item Integrar mecanismos de observabilidade para a recolha de métricas e \textit{logs} centralizados.
    \item Automatizar a configuração e o \textit{deploy} com recurso a Ansible e a uma \textit{pipeline} simples de CI/CD.
\end{itemize}

\section{Estrutura do Documento}
O presente relatório encontra-se organizado da seguinte forma: o Capítulo 2 apresenta o Estado da Arte das tecnologias envolvidas; o Capítulo 3 detalha o levantamento das necessidades funcionais e a escolha do ambiente base; o Capítulo 4 descreve a arquitetura e implementação da plataforma; o Capítulo 5 aborda os testes e resultados obtidos; por fim, o Capítulo 6 sintetiza as conclusões do trabalho.