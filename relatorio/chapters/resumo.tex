O presente projeto tem como objetivo principal a conceção e implementação de uma plataforma de serviços de rede gerida de forma reprodutível, recorrendo exclusivamente a ferramentas \textit{open source}. A infraestrutura desenvolvida engloba serviços fundamentais para uma rede de pequena a média dimensão, tais como servidor DNS, \textit{Reverse Proxy} HTTPS, serviços \textit{web}, e-mail, servidor de ficheiros e VPN. 

Em contraste com abordagens monolíticas tradicionais, todos os serviços são implementados em \textit{containers} (Docker/Podman) através de configuração declarativa, garantindo o versionamento e a reprodutibilidade do ambiente. O trabalho foca-se na aplicação rigorosa de boas práticas de segurança, incluindo o princípio do mínimo privilégio, \textit{hardening} de serviços e encriptação TLS. Adicionalmente, o projeto integra mecanismos avançados de observabilidade (para métricas e \textit{logs} centralizados) e a orquestração de \textit{deployments} através de ferramentas como Ansible e \textit{pipelines} de CI/CD. O resultado final demonstra como um pequeno \textit{data center} pode ser gerido de forma automatizada, ágil e resiliente a falhas.

\vspace{1em}
\noindent \textbf{Palavras-chave:} \textit{Containers}, Orquestração, Serviços de Rede, CI/CD, Observabilidade, \textit{Open Source}.